\setlength\tabcolsep{1 pt} % to reduce the margin of the table
\newcommand{\sep}{-0.1cm} % separation between elements of itemize
\newcommand{\sty}{\tt \scriptsize}
\renewcommand\labelitemi{--} % change the label of itemize to "-" 

\newcommand{\ins}{{\greentext{Instruction:}}}
\newcommand{\inp}{\newline{\greentext{Input:}}}

\begin{table*}[th!]
\centering
\small
\resizebox{1\textwidth}{!}{
\begin{tabular}{p{0.29\textwidth}|p{0.40\textwidth}|p{0.32\textwidth}|c}
\toprule
\multicolumn{1}{c}{\textbf{Instruction + Instance Input}}   & \multicolumn{1}{c}{\textbf{Target Response}} & \multicolumn{1}{c}{\textbf{\gptself{} Response}}   & \multicolumn{1}{c}{\textbf{Rating}}\\
\midrule
\sty
\ins{} Give students tips on how to keep their nerves under control during class presentations. &  
\sty
Everyone gets a little nervous when speaking in front of a group. That’s normal. But the more preparation you’ve done—meaning plenty of researching, organizing, and rehearsing—the more confident you’ll be. Preparation is the key. And if you make a mistake or stumble over your words, just regroup and keep going. Nobody’s perfect, and nobody expects you to be. & 
\vspace{-2mm}
\sty
\begin{itemize}[leftmargin=0.3cm, itemsep=\sep, topsep=0.1pt]
    \item Practice your presentation in front of friends and family. 
    \item Don't drink too much caffeine before the presentation. 
    \item Take deep breaths before you start speaking. 
    \item If you feel nervous, smile. 
    \item Remember that everyone in the audience is rooting for you. 
\end{itemize} & \sty A
\\
\midrule
\sty
\ins{} Make a list of the pros and cons of the given decision. 
\inp{} Implementing a Remote Working Policy 
& 
\sty
Pros: \begin{itemize}[leftmargin=0.3cm, itemsep=\sep, topsep=0.1pt]
    \item Allowing employees to work from home can reduce their stressful commutes and increase their productivity.
    \item Remote working provides greater flexibility, particularly for working parents and caregivers who have other responsibilities. 
    % \item It's possible to hire highly creative people who live further away if they don't have to come into the office every day.
    % \item People who aren't distracted by office noises and regular interruptions will perform better. 
    % \item Team members' communication skills will improve, as they will learn to communicate more effectively using email and IM, for example. 
    \item \dots (more points)
\end{itemize}

Cons:
\begin{itemize}[leftmargin=0.3cm, itemsep=\sep, topsep=0.1pt]
    \item It may be more difficult for team members to build meaningful, productive, creative relationships with one another.    
    \item \dots (more points)
    \end{itemize}
    &  
    \sty
    Pros:
\begin{itemize}[leftmargin=0.3cm, itemsep=\sep, topsep=0.1pt]
    \item Employees can work from home and save on commuting costs.     
    \item Employees can spend more time with their families.    
\end{itemize}
Cons:
\begin{itemize}[leftmargin=0.3cm, itemsep=\sep, topsep=0.1pt]   
    \item Employees may not be as productive as they would be in the office.    
    \item Employees may not be as motivated to work. 
\end{itemize} & \sty A
\\
\midrule
\sty
\ins{} A color description has been provided. Find the CSS code associated with that color. 
\inp{} A light red color with a medium light shade of pink & 
\sty
\#DB7093 ($\textcolor[HTML]{DB7093}\lgblksquare$)& 
\sty
~\#ff66cc ($\textcolor[HTML]{ff66cc}\lgblksquare$) & \sty B \\
%\yeganeh{Yizhong I don't think it's correct. the generated color is rose pink, while we are looking for a shade of pale red} \yizhong{yeah it seems we can rate this a level-3 if not 4?} \yeganeh{Yes}
\midrule

\sty
\ins{} Give some examples of what people usually say in the given social situation.  
\inp{} when someone arrives safely &
\sty
\vspace{-2mm}
\begin{itemize}[leftmargin=0.3cm, itemsep=\sep, topsep=0.1pt]
    \item ``Glad you made it safe and sound.''
    \item ``Good to see you got here unscathed.''
    \item ``So, bulletproof vests actually work!''
\end{itemize} 
&  
\sty
Thank God, I made it. & \sty B\\
\midrule
\sty
\ins{} You are given a sentence in first-order logic. Translate it into natural language.  
\inp{}
$\exists x \forall	y (soldier(x) $ 
$\to general(x,y))$ 
& 
\sty
There is someone (x) such that if he is a soldier, then he is general of everyone. &  
\sty
There exists a soldier such that for every general, he is a general. & \sty C\\
\midrule 
\sty 
\ins{}
Design an outline for a blog post based on the given information and list the sections accordingly. 
\inp{}
 This blog post is going to about making a list of last-minute gift ideas. Also, the tone of the post is going to be relaxed and casual. & 
\sty \begin{itemize}[leftmargin=0.3cm, itemsep=\sep, topsep=0 pt]
%\item Section: Gift card to a favorite restaurant or store
\item Section: Movie tickets
\item Section: A fun game console
%Section: Books 
%Section: Candles, flameless candles
%Section: Portable battery charger for phones/tablets
%Section: Water filter pitcher
\item Section: Personalized items with photos/artwork
\item ...(more sections)
%Section: Favorite team gear (hat, sweatshirt)
\item Takeaway: Don't stress about out running out of time to buy, make a gift. \end{itemize}&
\sty \begin{itemize}[leftmargin=0.3cm, itemsep=\sep, topsep=0 pt]
\item Introduction 
\item List of Gift Ideas
\item Conclusion \end{itemize}
& \sty C
 \\
\midrule
\sty
\ins{}
Write a Jira ticket for the given task. 
\inp{}
New Employee onboarding & 
\sty
Title: Please help our new employee get settled.
Prepare a desk, chair, and computer for them to use.
Create their company account, email, and temporary password.
Grant access to company documents. & 
\sty
Hi [Manager],
I’m writing to you because I have a question about the new employee onboarding process. I’ve been working here for a few months now, and I’m still not sure how it works. Could you please explain it to me? & \sty D\\
\midrule
\sty 
\ins{}
Create alliterations by finding synonyms for words in the given sentence. 
\inp{}
David wears a hat everyday. & \sty David dons a derby daily. & \sty  David wears a hat every day. & \sty D \\
\bottomrule
\end{tabular}
}
\caption{Examples in the User-oriented Instructions dataset (\S\ref{sec:user_instructions}) and predictions from \gptself{}. %\todo{Include more tasks (rating 1-4), sorted by rating.}
% \todo{explain what ratings are; add reference to the corresponding section.}
% \daniel{check this:}
The right column indicates one of the four quality ratings assigned to the model's response, with ``A'' indicating  ``valid and satisfying'' responses (highest)
and  ``D'' indicating ``irrelevant or invalid response'' (lowest).
}
\label{tab:case_study}
\end{table*}



\begin{comment}
\begin{table*}[th!]
\centering
\small
\resizebox{1\textwidth}{!}{
\begin{tabular}{p{0.15\textwidth}|p{0.13\textwidth}|p{0.33\textwidth}|p{0.3\textwidth}|c}
\toprule
\multicolumn{1}{c}{\textbf{Instruction}}  & \multicolumn{1}{c}{\textbf{Input}}  & \multicolumn{1}{c}{\textbf{Target}} & \multicolumn{1}{c}{\textbf{\gptself{} Response}}   & \multicolumn{1}{c}{\textbf{Rating}}\\
\midrule
\sty
Give students tips on how to keep their nerves under control during class presentations. & & 
\sty
Everyone gets a little nervous when speaking in front of a group. That’s normal. But the more preparation you’ve done—meaning plenty of researching, organizing, and rehearsing—the more confident you’ll be. Preparation is the key. And if you make a mistake or stumble over your words, just regroup and keep going. Nobody’s perfect, and nobody expects you to be. & 
\vspace{-2mm}
\sty
\begin{itemize}[leftmargin=0.3cm, itemsep=\sep, topsep=0.1pt]
    \item Practice your presentation in front of friends and family. 
    \item Don't drink too much caffeine before the presentation. 
    \item Take deep breaths before you start speaking. 
    \item If you feel nervous, smile. 
    \item Remember that everyone in the audience is rooting for you. 
\end{itemize} & \sty 4
\\
\midrule
\sty
Make a list of the pros and cons of the given decision. & 
\sty
Implementing a Remote Working Policy & 
\sty
Pros: \begin{itemize}[leftmargin=0.3cm, itemsep=\sep, topsep=0.1pt]
    \item Allowing employees to work from home can reduce their stressful commutes and increase their productivity.
    \item Remote working provides greater flexibility, particularly for working parents and caregivers who have other responsibilities. 
    % \item It's possible to hire highly creative people who live further away if they don't have to come into the office every day.
    % \item People who aren't distracted by office noises and regular interruptions will perform better. 
    % \item Team members' communication skills will improve, as they will learn to communicate more effectively using email and IM, for example. 
    \item \dots (more points)
\end{itemize}

Cons:
\begin{itemize}[leftmargin=0.3cm, itemsep=\sep, topsep=0.1pt]
    \item It may be more difficult for team members to build meaningful, productive, creative relationships with one another.    
    \item \dots (more points)
    \end{itemize}
    &  
    \sty
    Pros:
\begin{itemize}[leftmargin=0.3cm, itemsep=\sep, topsep=0.1pt]
    \item Employees can work from home and save on commuting costs.     
    \item Employees can spend more time with their families.    
\end{itemize}
Cons:
\begin{itemize}[leftmargin=0.3cm, itemsep=\sep, topsep=0.1pt]   
    \item Employees may not be as productive as they would be in the office.    
    \item Employees may not be as motivated to work. 
\end{itemize} & \sty 4
\\
\midrule
\sty
A color description has been provided. Find the CSS code associated with that color. & 
\sty
A light red color with a medium light shade of pink & 
\sty
\#DB7093 ($\textcolor[HTML]{DB7093}\lgblksquare$)& 
\sty
~\#ff66cc ($\textcolor[HTML]{ff66cc}\lgblksquare$) & \sty 3 \\
%\yeganeh{Yizhong I don't think it's correct. the generated color is rose pink, while we are looking for a shade of pale red} \yizhong{yeah it seems we can rate this a level-3 if not 4?} \yeganeh{Yes}
\midrule

\sty
Give some examples of what people usually say in the given social situation. & 
\sty
when someone arrives safely &
\sty
\vspace{-2mm}
\begin{itemize}[leftmargin=0.3cm, itemsep=\sep, topsep=0.1pt]
    \item ``Glad you made it safe and sound.''
    \item ``Good to see you got here unscathed.''
    \item ``So, bulletproof vests actually work!''
\end{itemize} 
&  
\sty
Thank God, I made it. & \sty 3\\
\midrule
\sty
You are given a sentence in first-order logic. Translate it into natural language. & 
\sty
$\exists x \forall	y (soldier(x) $ 
$\to general(x,y))$ 
& 
\sty
There is someone (x) such that if he is a soldier, then he is general of everyone. &  
\sty
There exists a soldier such that for every general, he is a general. & \sty 2\\
\midrule 
\sty Design an outline for a blog post based on the given information and list the sections accordingly. &
\sty This blog post is going to about making a list of last-minute gift ideas. Also, the tone of the post is going to be relaxed and casual. & 
\sty \begin{itemize}[leftmargin=0.3cm, itemsep=\sep, topsep=0 pt]
%\item Section: Gift card to a favorite restaurant or store
\item Section: Movie tickets
\item Section: A fun game console
%Section: Books 
%Section: Candles, flameless candles
%Section: Portable battery charger for phones/tablets
%Section: Water filter pitcher
\item Section: Personalized items with photos/artwork
\item ...(more sections)
%Section: Favorite team gear (hat, sweatshirt)
\item Takeaway: Don't stress about out running out of time to buy, make a gift. \end{itemize}&
\sty \begin{itemize}[leftmargin=0.3cm, itemsep=\sep, topsep=0 pt]
\item Introduction 
\item List of Gift Ideas
\item Conclusion \end{itemize}
& \sty 2
 \\
\midrule
\sty
Write a Jira ticket for the given task. & 
\sty
New Employee onboarding & 
\sty
Title: Please help our new employee get settled.
Prepare a desk, chair, and computer for them to use.
Create their company account, email, and temporary password.
Grant access to company documents. & 
\sty
Hi [Manager],
I’m writing to you because I have a question about the new employee onboarding process. I’ve been working here for a few months now, and I’m still not sure how it works. Could you please explain it to me? & \sty 1\\
\midrule
\sty Create alliterations by finding synonyms for words in the given sentence. & \sty David wears a hat everyday. & \sty David dons a derby daily. & \sty  David wears a hat every day. & \sty 1 \\
\bottomrule
\end{tabular}
}
\caption{Examples in the User-oriented Instructions dataset and predictions from \gptself{}. %\todo{Include more tasks (rating 1-4), sorted by rating.}
}
\label{tab:case_study}
\end{table*}
\end{comment}