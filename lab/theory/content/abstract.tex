The crystallization of modeling methods around the Transformer architecture has been a boon for practitioners. 
Simple, well-motivated architectural variations can transfer across tasks and scale, increasing the impact of modeling research. 
However, with the emergence of state-of-the-art 100B+ parameters models, large language models are increasingly expensive to accurately design and train. 
Notably, it can be difficult to evaluate how modeling decisions may impact emergent capabilities, given that these capabilities arise mainly from sheer scale alone.
In the process of building BLOOM--the Big Science Large Open-science Open-access Multilingual language model--our goal is to identify an architecture and training setup that makes the best use of our 1,000,000 A100-GPU-hours budget.
Specifically, we perform an ablation study at the billion-parameter scale comparing different modeling practices and their impact on zero-shot generalization.
% We perform all our experiments on 1.3B models, providing a compromise between compute costs and the likelihood that our conclusions will hold for the target 100B+ model. 
In addition, we study the impact of various popular pre-training corpora on zero-shot generalization. 
We also study the performance of a multilingual model and how it compares to the English-only one. 
Finally, we consider the scaling behaviour of Transformers to choose the target model size, shape, and training setup. All our models and code are open-sourced at \url{https://huggingface.co/bigscience}.