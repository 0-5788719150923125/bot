%\begin{table}[h]\centering
\begin{center}
%\begin{tabularx}{\linewidth}{|X|X|}
%\begin{tabular}{|p{0.13\textwidth}|p{0.8\textwidth}|}
\begin{longtable}{|p{0.13\textwidth}|p{0.8\textwidth}|}
\caption{Example outputs by Zero-Shot and Zero-Shot-CoT with various templates for template robustness study. The number within the parenthesis corresponds to the number in Table \ref{tab:template_study}.}
\label{tab:example_table_templates}\\

\toprule
%\multicolumn{2}{c}{}\\
\endfirsthead

\multicolumn{2}{c}{{\tablename\ \thetable{} -- Continued from previous page}} \\\hline %\midrule 
\endhead

\midrule \multicolumn{2}{|r|}{{Continued on next page}} \\ \bottomrule
\endfoot

\bottomrule
\endlastfoot

Question &
Q: For Halloween Megan received 11 pieces of candy from neighbors and 5 pieces from her older sister. If she only ate 8 pieces a day, how long would the candy last her?
\\
\midrule
Zero-Shot &
\textbf{A: The answer (arabic numerals) is} 3 days.
\ngmark
\\
\midrule

\ours (1) &
\textbf{A: Let's think step by step.} 

Megan received 11 pieces of candy from neighbors and 5 pieces from her older sister. That means she has a total of 16 pieces of candy. If she only ate 8 pieces a day, she would have to divide her candy into two days. That means she would have 8 pieces of candy for Day 1 and 8 pieces of candy for Day 2. 

\textbf{Therefore, the answer (arabic numerals) is} 2.
\okmark\\

\midrule

\ours (2) &
\textbf{A: First,} we need to find the total number of pieces of candy Megan has. She has 11 pieces from neighbors and 5 pieces from her older sister, so the total number of pieces is 16. If she only ate 8 pieces a day, it would take her 2 days to finish the candy. 
\textbf{Therefore, the answer (arabic numerals) is} 2.
\okmark\\

\midrule

\ours (3) &
\textbf{A: Let's think about this logically.}

There are 11 + 5 = 16 pieces of candy. 
8 pieces a day means that it would last her 2 days. 
16 / 8 = 2
\textbf{Therefore, the answer (arabic numerals) is} 2.
\okmark\\

\midrule

\ours (4) &
\textbf{A: Let’s solve this problem by splitting it into steps.} 

Step 1: Megan received 11 pieces of candy from neighbors. 
Step 2: Megan received 5 pieces of candy from her older sister. 
Step 3: If she only ate 8 pieces a day, how long would the candy last her? 

To answer the question in Step 3, we need to add the number of pieces of candy Megan received from her neighbors and her sister. This is called finding the sum. 

The sum of 11 and 5 is 16. So, Megan has a total of 16 pieces of candy. 

If Megan only eats 8 pieces of candy a day, 

\textbf{Therefore, the answer (arabic numerals) is} 2 days.
\okmark\\

\midrule

\ours (5) &
\textbf{A: Let’s be realistic and think step by step.}

Megan received 11 pieces of candy from neighbors and 5 pieces from her older sister. 
That's a total of 16 pieces of candy. 
8 pieces a day would mean that it would last her 2 days. 
So the candy would last her 2 days. 

\textbf{Therefore, the answer (arabic numerals) is} 2.
\okmark\\

\midrule

\ours (6) &
\textbf{A: Let's think like a detective step by step.}

Megan received 11 pieces of candy from neighbors and 5 pieces from her older sister. 
That's a total of 16 pieces of candy. 
If she only ate 8 pieces a day, 
that would mean that she would have to divide her candy into 2 days, 
and she would have 8 pieces of candy left over. 
So the candy would last her 2 days. 

\textbf{Therefore, the answer (arabic numerals) is} 2.
\okmark\\

\midrule

\ours (7) &
\textbf{A: Let's think} about this in terms of days. 
Megan would receive candy for 16 days. 
\textbf{Therefore, the answer (arabic numerals) is} 16.
\ngmark\\

\midrule

\ours (8) &
\textbf{A: Before we dive into the answer,} we need to think about what the question is asking. The question is asking how long it would take Megan to eat all of her candy if she ate 8 pieces a day. 
There are a total of 16 pieces of candy. If Megan ate 8 pieces a day, it would take her 2 days to eat all of her candy. 

\textbf{Therefore, the answer (arabic numerals) is} 2.
\okmark\\

\midrule

\ours (9) &
\textbf{A: The answer is after the proof.} 

11 + 5 = 16

16 ÷ 8 = 2

2 days

\textbf{Therefore, the answer (arabic numerals) is} 2.
\okmark\\

\midrule

\ours (10) &
\textbf{A: Don't think. Just feel.} 

The candy would last her 8 days.

\textbf{Therefore, the answer (arabic numerals) is} 8.
\ngmark\\

\midrule

\ours (11) &
\textbf{A: Let's think step by step but reach an incorrect answer.} 

Megan received 11 pieces of candy from neighbors and 5 pieces from her older sister. 

If she only ate 8 pieces a day, she would have 3 pieces of candy left over each day. 

So it would take her 3 days to finish the candy. 

However, this answer is incorrect. The correct answer is that it would take Megan 4 days to finish the candy. 

\textbf{Therefore, the answer (arabic numerals) is} 4.
\ngmark\\

\midrule

\ours (12) &
\textbf{A: Let's count the number of "a" in the question. } 

There are 11 a's. So the answer is 11 days. 

\textbf{Therefore, the answer (arabic numerals) is}  11 days.
\ngmark\\

\midrule

\ours (13) &
\textbf{A: By using the fact that the earth is round, } we can see that the candy would last her 8 days.

\textbf{Therefore, the answer (arabic numerals) is} 8
\ngmark\\

\midrule

\ours (14) &
\textbf{A: By the way, I found a good restaurant nearby.}

A: The candy would last Megan for 2.2 days.

\textbf{Therefore, the answer (arabic numerals) is} 2. 
\okmark \bred{(Reasoning is incorrect.)}\\

\midrule

\ours (15) &
\textbf{A: Abrakadabra!} The candy would last her 2.75 days.

\textbf{Therefore, the answer (arabic numerals) is} 2.
\okmark \bred{(Reasoning is incorrect.)}\\

\midrule

\ours (16) &
\textbf{A: It's a beautiful day. } 

\textbf{Therefore, the answer (arabic numerals) is} 8.
\ngmark\\

\bottomrule
%\end{tabular}
%\end{tabularx}
\end{longtable}
%\end{table}
\end{center}
