\section{Discussion and Conclusion}
\label{sec:conclusion}
Neuroimaging technologies, including functional magnetic resonance imaging (fMRI) are powerful noninvasive tools for examining the brain functioning. There is an emerging nation-wide interest in conducting neuroimaging studies for investigating the connection between the biology of the brain, and demographic variables and clinical outcomes such as mental disorders. Such studies provide an unprecedented opportunity for cross-cutting investigations that may offer new insights to the differences in brain function and organization across subpopulations in the society (such as biological sex and age groups) as well as reveal neurophysiological mechanisms underlying brain disorders (such as psychiatric illnesses and neurodegenerative diseases). These studies have a tremendous impact in social studies and biomedical sciences. For example, mental disorders are the leading cause of disability in the USA and roughly 1 in 17 have a seriously debilitating mental illness. To address this burden, national institutions such as NIH have included brain-behavior research as one of their strategic objectives and stated that sound efforts must be made to redefine mental disorders into dimensions or components of observable behaviors that are more closely aligned with the biology of the brain. Using brain imaging data to predict diagnosis has great potential to result in mechanisms that target for more effective preemption and treatment.

In this paper, we present \methodfull, a specialized graph Transformer model with \pooling for brain network analysis. Extensive experiments on two large-scale brain network datasets demonstrate that our \methodtable achieves superior performance over SOTA baselines of various types. Specifically, to model the potential node feature similarity in brain networks, we design \poolingshort and prove its effectiveness both theoretically and empirically. Lastly, the re-standardized dataset split for ABIDE can provide a fair evaluation for new methods in the community. For future work, \methodtable can be improved with explicit explanation modules and used as the backbone for further brain network analysis, such as digging essential neural circuits for mental disorders and understanding cognitive development in adolescents.

