\begin{center}
\begin{longtable}{|p{0.25\textwidth}|p{0.40\textwidth}|p{0.25\textwidth}|}
\caption{Example-based comparison between \ours and \theirs from MultiArith.}
\label{tab:example_multiarith_comparison}\\

\toprule \multicolumn{1}{|c|}{\textbf{Question}} & \multicolumn{1}{c|}{\textbf{\ours}} & \multicolumn{1}{c|}{\textbf{\theirs}} \\ \midrule 
\endfirsthead

\multicolumn{3}{c}%
{{\tablename\ \thetable{} -- continued from previous page}} \\
\toprule \multicolumn{1}{|c|}{\textbf{Question}} & \multicolumn{1}{c|}{\textbf{\ours}} & \multicolumn{1}{c|}{\textbf{\theirs}} \\ \midrule
\endhead

\multicolumn{3}{|r|}{{Continued on next page}} \\ \bottomrule
\endfoot

\bottomrule
\endlastfoot

%Q: A pet store had 102 puppies. In one day they sold 21 of them and put the rest into cages with 9 in each cage. How many cages did they use?
%&A \bred{(PREDICTION IS CORRECT BUT REASONING IS WRONG)}: Let's think step by step. 
%There are 102 puppies. 21 of them were sold, so that means there are 81 puppies left. 81 is not divisible by 9, so that means there is 1 puppy left over, and the rest can go into cages with 9 puppies in each. That means there are 9 cages in total. Therefore, the answer (arabic numerals) is 9. \okmark
%&A: Jerry started with 34 books. A librarian took 7 of them, so he was left with 34 - 7 = 27. He can put 3 books on a shelf, so he will need 27 / 3 = 9 shelves. The answer is 9. \okmark \\

%\midrule

%Q: Victor was helping the cafeteria workers pick up lunch trays, but he could only carry 7 trays at a time. If he had to pick up 23 trays from one table and 5 trays from another, how many trips will he make?
%&A \bred{(PREDICTION IS CORRECT BUT REASONING IS WRONG)}: Let's think step by step. 
%Victor can carry 7 trays at a time. If he has to pick up 23 trays from one table, he will need to make 3 trips. If he has to pick up 5 trays from another table, he will need to make 1 trip. Therefore, he will need to make 4 trips in total. Therefore, the answer (arabic numerals) is 4. \okmark
%&A \bred{(PREDICTION IS CORRECT BUT REASONING IS WRONG)}: Victor can carry 7 trays at a time. If he has to pick up 23 trays from one table and 5 trays from another, he will have to make 23/7 = 3 trips from the first table, and 5/7 = 1 trip from the second table. So he will make 3 + 1 = 4 trips in total. The answer is 4.\okmark \\

%\midrule

Q: At the fair Adam bought 13 tickets. After riding the ferris wheel he had 4 tickets left. If each ticket cost 9 dollars, how much money did Adam spend riding the ferris wheel?
&A \bred{(ONE UNNECESSARY STEP ERROR)}: Let's think step by step. 
Adam bought 13 tickets. 
That means he spent 13 * 9 = 117 dollars on tickets. 
After riding the ferris wheel, he had 4 tickets left. 
That means he used 9 tickets to ride the ferris wheel. 
That means he spent 9 * 9 = 81 dollars on riding the ferris wheel. 

So the answer is 117 - 81 = 36 dollars. 
Therefore, the answer (arabic numerals) is 36. \ngmark
&A: Adam started with 13 tickets. He had 4 left after riding the ferris wheel. So he must have spent 13 - 4 = 9 tickets. 9 tickets times 9 dollars per ticket is 9 * 9 = 81. The answer is 81. \okmark \\

\midrule

Q: At a restaurant each adult meal costs \$5 and kids eat free. If a group of 15 people came in and 8 were kids, how much would it cost for the group to eat?
&A \bred{(ONE UNNECESSARY STEP ERROR)}: Let's think step by step. 
There are 8 kids, so that means there are 7 adults. 
Each adult meal costs \$5, so that means that the 7 adults will cost \$5*7=\$35 
Now we just need to add the two numbers together. 
\$35+8=43\$
So it would cost \$43 for the group to eat. 
Therefore, the answer (arabic numerals) is 43. \ngmark
&A: If 8 were kids, then that means there were 15 - 8 = 7 adults. Each adult meal costs \$5. So the total cost would be 7 * 5 = 35. The answer is 35.\okmark \\

\midrule

%Q: For Halloween Megan received 11 pieces of candy from neighbors and 5 pieces from her older sister. If she only ate 8 pieces a day, how long would the candy last her? &A \bblue{(CORRECT)}: Let's think step by step. Megan received 11 pieces of candy from neighbors and 5 pieces from her older sister. That means she has a total of 16 pieces of candy. If she only ate 8 pieces a day, she would have to divide her candy into two days. That means she would have 8 pieces of candy for Day 1 and 8 pieces of candy for Day 2. Therefore, the answer (arabic numerals) is 2. &A \bblue{(CORRECT)}: Megan received 11 pieces of candy from neighbors and 5 pieces from her older sister. So she had 11 + 5 = 16 pieces of candy. If she ate 8 pieces a day, then she would have 16 / 8 = 2 days worth of candy. The answer is 2. \\

%\midrule

%Q: Edward and his friend were buying trick decks from the magic shop for 6 dollars each. How much did they spend if Edward bought 3 decks and his friend bought 6 decks? &A \bblue{(CORRECT)}: Let's think step by step. Edward buys 3 decks. That's 3 × 6 = 18 dollars. His friend buys 6 decks. That's 6 × 6 = 36 dollars. So they spent 18 + 36 = 54 dollars. Therefore, the answer (arabic numerals) is 54. &A \bblue{(CORRECT)}: Edward bought 3 decks. His friend bought 6 decks. So in total they bought 3 + 6 = 9 decks. 9 decks at 6 dollars each is 9 * 6 = 54 dollars. The answer is 54. \\

%\midrule

%Q: A pet shelter had 3 puppies when another 3 were brought in. If 3 puppies a day are adopted, how long would it take for all of them to be adopted? &A \bred{(COMPLICATED REASONING ERROR)}: Let's think step by step. There are 3 puppies. 3 more puppies are brought in. That means there are now 6 puppies. 3 puppies are adopted. That means there are now 3 puppies. 3 more puppies are brought in. That means there are now 6 puppies. 3 puppies are adopted. That means there are now 3 puppies. 3 more puppies are brought in. That means there are now 6 puppies. 3 puppies are adopted. That means there are now 3 puppies. 3 more puppies are brought in. That means Therefore, the answer (arabic numerals) is 9. &A \bblue{(CORRECT)}: The shelter started with 3 puppies. Another 3 were brought in. So in total there are 3 + 3 = 6 puppies. If 3 puppies are adopted each day, then it will take 6 / 3 = 2 days for all of them to be adopted. The answer is 2. \\

%\midrule

%Q: Paige's team won their dodgeball game and scored 41 points total. If Paige scored 11 of the points and everyone else scored 6 points each, how many players were on her team? &A (COMPLICATED REASONING ERROR): Let's think step by step. We know that Paige scored 11 points and everyone else scored 6 points each. We also know that the total score was 41 points. We can set up the equation like this: x + 6 + 6 + ... + 6 = 41 x = 11 There are 11 players on Paige's team. Therefore, the answer (arabic numerals) is 11. &A (CORRECT; CoT is NG): Paige's team scored 41 points. Paige scored 11 of those points. So if everyone else scored 6 points each, then that is 6 * (number of players - 1). 6 * (number of players - 1) = 41 - 11. 6 * (number of players - 1) = 30. 30 / 6 = 5. So there were 5 players on her team. The answer is 5. \\

%\midrule

%Q: Maria was making baggies of cookies with 3 cookies in each bag. If she had 2 chocolate chip cookies and 16 oatmeal cookies, how many baggies could she make? &A (COMPLICATED REASONING ERROR): Let's think step by step. There are 3 cookies in each bag. There are 2 chocolate chip cookies. That means there is 1 more oatmeal cookie than there are chocolate chip cookies. That means there is 1 oatmeal cookie left over. That means she can make 5 baggies. Therefore, the answer (arabic numerals) is 5. &A (LOGICAL ERROR; ONE STEP MISSING): Maria was making baggies of 3 cookies each. She had 2 chocolate chip cookies and 16 oatmeal cookies. So she could make 2 + 16 = 18 baggies. The answer is 18. \\

%\midrule

Q: April's discount flowers was having a sale where each flower was 6 dollars. If Katie bought 5 roses and 5 daisies, how much did she spend? 
&A: Let's think step by step. First, we need to calculate how much each type of flower costs. There are 5 roses, and each rose costs 6 dollars. So, the cost of the roses is 5*6=30 dollars. There are 5 daisies, and each daisy costs 6 dollars. So, the cost of the daisies is 5*6=30 dollars. Then, we need to calculate the total cost. The total cost is the cost of the roses plus the cost of the daisies. So, the total cost is 30+30=60 dollars. Therefore, the answer (arabic numerals) is 60. \okmark
&A \bred{(CALCULATOR ERROR)}: Each flower was 6 dollars. Katie bought 5 roses and 5 daisies. So she spent 5 * 6 + 5 * 6 = 60 + 30 = 90. The answer is 90. \ngmark \\

\midrule

Q: While shopping for music online, Faye bought 2 country albums and 3 pop albums. Each album came with a lyric sheet and had 6 songs. How many songs did Faye buy total? &A: Let's think step by step. Faye bought 2 country albums. Each album has 6 songs. So she bought 2 * 6 = 12 songs from the country albums. Faye bought 3 pop albums. Each album has 6 songs. So she bought 3 * 6 = 18 songs from the pop albums. In total, Faye bought 12 + 18 = 30 songs. Therefore, the answer (arabic numerals) is 30. \okmark
&A \bred{(CALCULATOR ERROR)}: Faye bought 2 country albums and 3 pop albums. Each album had 6 songs. So she bought 2 * 6 + 3 * 6 = 24 songs. The answer is 24. \ngmark \\

\midrule

Q: Jerry was helping the cafeteria workers pick up lunch trays, but he could only carry 8 trays at a time. If he had to pick up 9 trays from one table and 7 trays from another, how many trips will he make?
&A \bred{(COMMONSENSE MISTAKE)}: Let's think step by step. 
Jerry can carry 8 trays at a time. 
There are 9 trays on the first table. 
That means Jerry will need to make 2 trips to the first table. 
There are 7 trays on the second table. 
That means Jerry will need to make 1 trip to the second table. 
In total, Jerry will make 3 trips. 
Therefore, the answer (arabic numerals) is 3. \ngmark
&A: Jerry can carry 8 trays at a time. If he has to pick up 9 trays from one table and 7 trays from another, he will have to make 2 trips. The answer is 2. \okmark \\

\midrule

Q: Kaleb bought 14 boxes of chocolate candy and gave 5 to his little brother. If each box has 6 pieces inside it, how many pieces did Kaleb still have?
&A: Let's think step by step. Kaleb bought 14 boxes of chocolate candy. This means that he had 84 pieces of chocolate candy. He gave 5 boxes to his little brother, which means that he gave his little brother 30 pieces of chocolate candy. Kaleb still had 54 pieces of chocolate candy. Therefore, the answer (arabic numerals) is 54. \okmark
&A \bred{(COMMONSENSE MISTAKE)}: Kaleb bought 14 boxes. Each box has 6 pieces. So 14 * 6 = 84 pieces. He gave 5 to his brother. So he has 84 - 5 = 79 pieces. The answer is 79. \ngmark \\

\midrule

Q: At the fair there were 12 people in line for the bumper cars. If 10 of them got tired of waiting and left and 15 more got in line, how many people would be in line? 
&A \bred{(OTHERS; REASONING IS NOT STARTED)}: Let's think step by step. There were 12 people in line for the bumper cars. 10 of them got tired of waiting and left. 15 more got in line. That means that there are now 15 people in line for the bumper cars. Therefore, the answer (arabic numerals) is 15. \ngmark
&A: There were originally 12 people in line. 10 of them left, so that left 12 - 10 = 2. Then 15 more got in line, so that is 2 + 15 = 17. The answer is 17. \okmark \\

\midrule

Q: Luke made 9 dollars mowing lawns and 18 dollars weed eating. If he only spent 3 dollar a week, how long would the money last him? 
&A \bred{(OTHERS; QUESTION UNDERSTANDING ERROR)}: Let's think step by step. How much money does Luke make in a week? How much money does Luke spend in a week? How much money does Luke have left over at the end of the week? How long will it take Luke to save up \$54? Luke makes \$27 a week. He spends \$3 a week. He has \$24 left over at the end of the week. It will take Luke 2 weeks to save up \$54. Therefore, the answer (arabic numerals) is 2. \ngmark
&A: Luke made 9 dollars mowing lawns and 18 dollars weed eating. In total, he made 9 + 18 = 27 dollars. If he spends 3 dollars a week, that is 3 dollars * 4 weeks = 12 dollars a month. So the money would last him 27 / 12 = 2.25 months. The answer is 2.25 months. \ngmark \myspace \bblue{(GT : 9 weeks)}\\

\midrule

Q: Each chocolate bar in a box cost \$3. If a box had 9 bars total and Wendy sold all but 3 bars, how much money would she have made? 
&A \bred{(OTHERS; MAXIMUM LIMIT OF TEXT LENGTH)}: Let's think step by step. Each chocolate bar in a box costs \$3. If a box had 9 bars total, then Wendy sold all but 3 bars. How much money would she have made? We can solve this problem using algebra. Let's start by creating a variable to represent the number of chocolate bars Wendy sold. We'll call this variable "x". If Wendy sold "x" chocolate bars, then she would have 9 - x chocolate bars left in the box. We know that each chocolate bar costs \$3, so the total cost of the chocolate bars Wendy sold \par
Therefore, the answer (arabic numerals) is: Wendy would have made \$12. \ngmark
&A: Each chocolate bar cost 3 dollars. So if Wendy sold all but 3, she would have sold 9 - 3 = 6. 6 * 3 = 18. The answer is 18. \okmark\\

%\midrule

%Q: Kaleb bought 14 boxes of chocolate candy and gave 5 to his little brother. If each box has 6 pieces inside it, how many pieces did Kaleb still have? &A: Let's think step by step. Kaleb bought 14 boxes of chocolate candy. This means that he had 84 pieces of chocolate candy. He gave 5 boxes to his little brother, which means that he gave his little brother 30 pieces of chocolate candy. Kaleb still had 54 pieces of chocolate candy. Therefore, the answer (arabic numerals) is 54. &A: Kaleb bought 14 boxes. Each box has 6 pieces. So 14 * 6 = 84 pieces. He gave 5 to his brother. So he has 84 - 5 = 79 pieces. The answer is 79. \\

%\midrule

%Q: Frank made 5 dollars mowing lawns and 58 dollars weed eating. If he only spent 7 dollar a week, how long would the money last him? &A: Let's think step by step. Frank made 5 dollars mowing lawns. He made 58 dollars weed eating. That means he made a total of 63 dollars. If he spends 7 dollars a week, that means he has 63-7=56 dollars left. If he spends 7 dollars a week, that means he can spend 56/7=8 weeks. Therefore, the money would last him 8 weeks. Therefore, the answer (arabic numerals) is 8. &A: Frank made 5 + 58 = 63 dollars. If he spends 7 dollars a week, that is 7 * 4 = 28 dollars a month. 63 / 28 = 2.25. The answer is 2.25 months. \\

%\midrule

\end{longtable}
\end{center}